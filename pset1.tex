\documentclass[11pt]{article}
\usepackage{graphicx}
\usepackage{hyperref}
\usepackage{fullpage}
\usepackage{amsfonts}
\usepackage{amssymb}
\usepackage{amsmath}
\usepackage{xcolor}
\usepackage{algorithm}
\usepackage{algorithmic}
\usepackage{enumitem}
\usepackage{fourier}
\usepackage[normalem]{ulem}


\newcommand{\F}{\mathbb{F}}
\newcommand{\np}{\mathop{\rm NP}}


\setlength{\parskip}{\medskipamount}
\setlength{\parindent}{0in}

\begin{document}


        \section*{Econ 1759: Pset 1}
        \textit{Cory Zimmerman}

\textbf{1.1}
\begin{itemize}
  \item Assets: $0.2 \cdot 0 + 0.2 \cdot 70 + 0.2 \cdot 80 + 0.4 \cdot 100 = 70$.
  \item Senior debt: $0.2 \cdot 0 + 0.8 \cdot 70 = 56$. 
  \item Sub debt: $0.4 \cdot 0 + \0.2 \cdot 10 + 0.4 \cdot 20 = 10$.
  \item Equity: $0.6 \cdot 0 + 0.4 \cdot 10 = 4$.
\end{itemize}

\textbf{1.2}
Suppose the bank raises $n < 10$ units of equity. 
\begin{itemize}
  \item Senior debt: $0.2n + 0.8 \cdot 70 = 56 + 0.2n$. 
  \item Sub debt: $0.2 \cdot 0 + 0.2n + 0.2 \cdot (10 + n) + 0.4 \cdot 20 = 10 + 0.4n$
  \item Equity: $0.6 \cdot 0 + 0.4 \cdot (10 + n) = 4 + 0.4n$.
\end{itemize}

$60$ percent of the value accrues to debt, and $40$ percent accrues to equity. Say $n = 8$. Senior debt is now worth $57.6$, sub debt is worth $13.2$, and equity is worth $7.2$. In this example, equity gets an increase of 3.2 for an investment of 8, which is a bad deal.

\textbf{1.3}
TO DO

\end{document}
